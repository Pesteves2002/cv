%%%%%%%%%%%%%%%%%
% This is an example CV created using altacv.cls (v1.6.4, 13 Nov 2021) written by
% LianTze Lim (liantze@gmail.com), based on the
% Cv created by BusinessInsider at http://www.businessinsider.my/a-sample-resume-for-marissa-mayer-2016-7/?r=US&IR=T
%
%% It may be distributed and/or modified under the
%% conditions of the LaTeX Project Public License, either version 1.3
%% of this license or (at your option) any later version.
%% The latest version of this license is in
%%    http://www.latex-project.org/lppl.txt
%% and version 1.3 or later is part of all distributions of LaTeX
%% version 2003/12/01 or later.
%%%%%%%%%%%%%%%%

%% Use the "normalphoto" option if you want a normal photo instead of cropped to a circle
% \documentclass[10pt,a4paper,normalphoto]{altacv}

\documentclass[10pt,a4paper,ragged2e,withhyper]{altacv}

%% AltaCV uses the fontawesome5 package.
%% See http://texdoc.net/pkg/fontawesome5 for full list of symbols.

% Change the page layout if you need to
\geometry{left=1.25cm,right=1.25cm,top=1.5cm,bottom=1.5cm,columnsep=1.2cm}

% The paracol package lets you typeset columns of text in parallel
\usepackage{paracol}


% Change the font if you want to, depending on whether
% you're using pdflatex or xelatex/lualatex
\ifxetexorluatex
  % If using xelatex or lualatex:
  \setmainfont{Lato}
\else
  % If using pdflatex:
  \usepackage[default]{lato}
\fi

% Change the colours if you want to
\definecolor{VividPurple}{HTML}{73A9C2}
\definecolor{SlateGrey}{HTML}{004953}
\definecolor{LightGrey}{HTML}{666666}
% \colorlet{name}{black}
% \colorlet{tagline}{PastelRed}
\colorlet{heading}{VividPurple}
\colorlet{headingrule}{VividPurple}
% \colorlet{subheading}{PastelRed}
\colorlet{emphasis}{SlateGrey}
\colorlet{body}{LightGrey}

% Change some fonts, if necessary
% \renewcommand{\namefont}{\Huge\rmfamily\bfseries}
% \renewcommand{\personalinfofont}{\footnotesize}
% \renewcommand{\cvsectionfont}{\LARGE\rmfamily\bfseries}
% \renewcommand{\cvsubsectionfont}{\large\bfseries}

% Change the bullets for itemize and rating marker
% for \cvskill if you want to
\renewcommand{\itemmarker}{{\small\textbullet}}
\renewcommand{\ratingmarker}{\faCircle}

%% Use (and optionally edit if necessary) this .tex if you
%% want to use an author-year reference style like APA(6)
%% for your publication list
\input{pubs-authoryear}

%% Use (and optionally edit if necessary) this .tex if you
%% want an originally numerical reference style like IEEE
%% for your publication list
% \input{pubs-num}

%% sample.bib contains your publications
\addbibresource{sample.bib}

\begin{document}
\name{Tomás Esteves}
\tagline{Studying Computer Science and Engineering at IST}
% Cropped to square from https://en.wikipedia.org/wiki/Marissa_Mayer#/media/File:Marissa_Mayer_May_2014_(cropped).jpg, CC-BY 2.0
%% You can add multiple photos on the left or right
% \photoL{2cm}{Yacht_High,Suitcase_High}
\personalinfo{%
  % Not all of these are required!
  % You can add your own with \printinfo{symbol}{detail}
  \email{tomasesteves2002@gmail.com}
%   \phone{000-00-0000}
  
  \location{Lisbon, PT}
  
   \github{Pesteves2002} % I'm just making this up though.
   
   \NewInfoField*{Strava}{\faStrava}{\strava}
   \Strava{Tomás Esteves}{https://www.strava.com/athletes/26750651}
   
%   \orcid{0000-0000-0000-0000} % Obviously making this up too.
  %% You can add your own arbitrary detail with
  %% \printinfo{symbol}{detail}[optional hyperlink prefix]
  % \printinfo{\faPaw}{Hey ho!}
  %% Or you can declare your own field with
  %% \NewInfoFiled{fieldname}{symbol}[optional hyperlink prefix] and use it:
  % \NewInfoField{gitlab}{\faGitlab}[https://gitlab.com/]
  % \gitlab{your_id}{https://www.strava.com/athletes/26750651}
	%%
  %% For services and platforms like Mastodon where there isn't a
  %% straightforward relation between the user ID/nickname and the hyperlink,
  %% you can use \printinfo directly e.g.
  % \printinfo{\faMastodon}{@username@instace}[https://instance.url/@username]
  %% But if you absolutely want to create new dedicated info fields for
  %% such platforms, then use \NewInfoField* with a star:
  % \NewInfoField*{mastodon}{\faMastodon}
  %% then you can use \mastodon, with TWO arguments where the 2nd argument is
  %% the full hyperlink.
  % \mastodon{@username@instance}{https://instance.url/@username}
}

\makecvheader

\NewInfoField*{Gitrepo}{\faGithub}{\gitrepo}

%% Depending on your tastes, you may want to make fonts of itemize environments slightly smaller
\AtBeginEnvironment{itemize}{\small}

%% Set the left/right column width ratio to 6:4.
\columnratio{0.6}

% Start a 2-column paracol. Both the left and right columns will automatically
% break across pages if things get too long.
\begin{paracol}{2}

\cvsection{Experience}

Learner at the Odin Project

\begin{itemize}
\item Finished the "Foundations" module 
\item Working on the "Full Stack Javascript" module
\end{itemize}

 \Gitrepo{Odin Project Repository}{https://github.com/Pesteves2002/Odin-Project}

\divider

Maintainer at Resumos LEIC

\Gitrepo{Resumos LEIC Repository}{https://github.com/Pesteves2002/resumos-leic}

\divider

Participant at Advent of Code (AoC)

\Gitrepo{AoC Repository}{https://github.com/Pesteves2002/AOC-2021}

\cvsection{Education}

\cvevent{Computer Science and Engineering}{Instituto Superior Técnico}{Sept 2020 -- June 2023 (Expected)}{}

\begin{itemize}
\item Finished Year 2 of Computer Science and Engineering
\item Mentor of freshman students
\end{itemize}

\divider

\cvevent{Finished Secondary School}{Escola Secundária do Restelo}{June 2020}{}
\begin{itemize}

\item Finished with an average of 19 out of 20
\item Participant at Regional "Olímpiadas da Física"
\item Participant at Regional "Olímpiadas da Matemática"
\item Participant at National "Corta-Mato Escolar"

\end{itemize}

\cvsection{Running Career}

Federated Runner at NucleOeiras (ADNO)

\begin{itemize}

\item First Place at National Mountain Running 2021 (Team)
\item Second Place at National Mountain Running 2021 (Individual)
\item Second Place at National Road Running 2019 (Team)
\item Third Place at National Cross Country Running 2018 (Team)
\item Top 10 at Marginal à Noite 2022 

\end{itemize}





% \divider

% \cvevent{Product Engineer}{Google}{23 June 1999 -- 2001}{Palo Alto, CA}

% \begin{itemize}
% \item Joined the company as employe \#20 and female employee \#1
% \item Developed targeted advertisement in order to use user's search queries and show them related ads
% \end{itemize}


\cvsection{Most Proud of}

\cvachievement{\faWrench}{Built my own PC}{Learned hardware and installed both Windows and Linux in Dual Boot}

\divider

\cvachievement{\faTrophy}{My 10km PR}{My time for 10km is 35:36 and I want to go sub 35 and even sub 34 by the end of 2022}



%% Switch to the right column. This will now automatically move to the second
%% page if the content is too long.
\switchcolumn

\cvsection{Strengths}

\cvtag{Hard-working}
\cvtag{Persuasive}\\
\cvtag{Motivated}

\cvsection{Languages}

	\vspace{1.2mm}
	\small \textbf{Portuguese} \hfill {Native \hspace{3pt}} \\
	
	\vspace{1mm}
	\small \textbf{English} \hfill{B1\hspace{3pt}}
	\vspace{1.2mm}
	
	
	
% \divider

\cvsection{Programming Languages}

\cvskill{Python}{4}
% \divider
\cvskill{C++}{3}
% \divider
\cvskill{C}{3}
% \divider
\cvskill{PostgreSQL}{3}
% \divider
\cvskill{HTML}{3}
% \divider
\cvskill{CSS}{3}
% \divider
\cvskill{Javascript}{3}
% \divider
\cvskill{Java}{2}


\cvsection{Tools}

\vspace{1.2mm}

Experience with version control systems, such as Git (GitHub) and CVS.\\
Experience with LaTeX.\\
Experience with Visual Studio Code and familiar with vim.

\vspace{1.2mm}

\cvsection{Operating Systems}

\vspace{1.2mm}
Using Linux as my daily driver.\\
Also comfortable using Windows.
\vspace{1.2mm}

\cvsection{A Day In My Life}

\vspace{1.2mm}

% Adapted from @Jake's answer from http://tex.stackexchange.com/a/82729/226
% \wheelchart{outer radius}{inner radius}{
% comma-separated list of value/text width/color/detail}
% Some ad-hoc tweaking to adjust the labels so that they don't overlap
\hspace*{-10em}  %% quick hack to move the wheelchart a bit left
\wheelchart{1.0cm}{0.5cm}{%
  30/13em/accent!30/Sleeping,
  20/9em/accent!60/Running,
  10/11em/accent!40/Spending time with family and friends,
  25/8em/accent!20/\footnotesiz Studying,
  15/9em/accent/Working on my own projects}


\end{paracol}

\end{document}
